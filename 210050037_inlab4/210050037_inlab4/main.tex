\documentclass{beamer}
\usepackage{hyperref}
\usepackage[utf8]{inputenc}

\usetheme{Madrid}
\usecolortheme{default}

\title[Latex basics \& Advanced] 
{CS 251- Lab4 \LaTeX Basics \& Advanced}


\author[B V S S Prabandh] 
{B V S S Prabandh}

\institute[IITB] 
{
  IIT Bombay
}

\date[2022]
{August 2022}


\begin{document}

\frame{\titlepage}

\begin{frame}
\frametitle{Introduction of myself}
Hello, I am Prabandh.This is my first Latex project for CS 251. I study at IITB  (sophiemore). I am from 
Andhra Pradesh in India.My hoobies are gaming and streaming.Thats all about me.I am so excited to be here. :).
\begin{figure}[htp]
    \centering
    \includegraphics[width=4cm]{iitb.png}
    \caption{IIT logo}
    \label{fig:school}
\end{figure}
\end{frame}
\begin{frame}
\frametitle{Table of Contents}
\tableofcontents
\end{frame}
\section{Introduction}
\begin{frame}
\frametitle{Introduction}
  We first see the power of frames in \textbf{\LaTeX}. We dont need to write each
and every slide just for a new line.
\pause
    We can just use beamer class with the feature of pauses.
\pause
    However, \textbf{\LaTeX} has another ( rather the most important usage ), namely the use \textcolor{red}{formatting text} in a more mathematical way.
\end{frame}
\section{Equations}
\begin{frame}
\frametitle{Equations}
 We can write many equations, can be labelled like the following
\begin{equation}
e^{i\alpha}=cos(\alpha)+isin(\alpha) 
\end{equation}
\pause
    or the unlabelled equations like the force between two charges given by
    \[F=\frac{1}{4\pi\epsilon}\frac{q_1q_2}{r^2}\]
\end{frame}
\section{Itemize and Linking}
\begin{frame}
\frametitle{Itemize and Linking}
Also, \textbf{\LaTeX} can be used to present the items in a list format, for example,
some common ways of sorting an array are:
\begin{itemize}
    \item Bubble Sort
    \item Insertion Sort
    \pause 
       , then there are the more rigorous algorithms like
    \item QuickSort
    \item Heap Sort
    \pause 
      , then the best known algorithm
    \item \textcolor{red}{Monkey sort} or Bogo-sort.
\end{itemize}
Some pointers to the last algorithm can be found at \href{https://en.wikipedia.org/wiki/Bogosort}{\textcolor{blue}{here}}.
\end{frame}
\section{Matrices}
\begin{frame}
\frametitle{Matrices}
We can also write matrices in \textbf{\LaTeX}, for example the identity matrix of size
(3x3) is
\begin{center}
\begin{eq}
 I_3 = \begin{bmatrix}
1 & 0 & 0\\
0 & 1 & 0\\
0 & 0 & 1
\end{bmatrix}  
\end{eq}
\end{center}
\pause
    \textcolor{red}{Bonus: try to indent like the below equation}
\begin{equation}
\begin{split}
(\textbf{a}.\textbf{b})^2 & = (\sum a_ib_i)^2\\
                        & \leq (\sum a_i^2)(\sum b_i^2)
\end{split}
\end{equation}
\end{frame}
\end{document}
